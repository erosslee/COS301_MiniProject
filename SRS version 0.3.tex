\documentclass[11pt,a4paper]{article}
\usepackage{hyperref}
\begin{document}
\begin{titlepage}
\title{\textbf{Software Requirements Specification}\\Computer Science Marks System\\ \small Version: 0.3\\ \url{https://github.com/erosslee/COS301_MiniProject}}

\author{\textit{By group 10:} \\Estian Rosslee 12223426\\Moeletji Semenya 12349136\\Thulasizwe Mavuso 29236259\\Uteshlen Nadesan 28163304\\Zenadia Groenewald 12265676\\Z\"uhnja Riekert 12040593\\ \\
\textit{For:} \\Mr. Jan Kroeze (University of Pretoria)}



\maketitle
\end{titlepage}
\pagebreak
\tableofcontents
\pagebreak
\section{Introduction}
\textbf{(NOT DONE Zenadia said she'll do this part)}\\ \\


\section{Vision}
\textbf{(DONE by Z\"uhnja - needs to be checked)}\\ \\
The main purpose of this project is that marks can easily be added directly to the university\textquoteright
s database through a user-friendly interface. No in between procedures.   Evaluators will be able to access the system with a mobile approach or via computer while they are assessing.  Students will be able to log into the system and view all of their marks. \\ \\
The client\textquoteright
s main aim with this project is to simplify and increase the efficiency of the marking system.

\section{Background}
\textbf{(DONE by Z\"uhnja - needs to be checked)}\\ \\
Currently students view their marks for COS modules once it is posted on the CS-site.  It is usually listed according to their student numbers.  Thus students are able to view each other\textquoteright s marks.  \\ \\
During COS practical evaluations, markers evaluate each student\textquoteright s work and write their marks down on paper.  That then has to be passed to someone who has the time to enter the marks into the University\textquoteright s system.  In the process of this marking system the marks can easily get lost. \\ \\
This project eliminates these threats.  Marks won\textquoteright t have the opportunity to get lost, since it will be added to the system immediately. Students will be able to immediately access their marks once it is added to the system and they won\textquoteright t be able to view any marks that are not their own.  \\ \\

\section{Stakeholders}
\textbf{(NOT DONE Who will do this?)}\\ \\
\section{Architectural requirements}
\textbf{(DONE by Thulasiswe - needs to be checked)}\\ \\
\subsection{Access channel requirements}
The system's services are to be accessed by Android application clients and Browser clients through its web interface. These are the only two platforms that are to be supported by the system.
\subsection{Quality requirements}
\subsubsection{Performance}
	\begin{itemize}
	\item System must not take more than 1.5 seconds to return a searched for student.
	\item A request to view marks must be granted within 1.2 seconds.
	\item System must have auto complete mechanism for search services.
	\end{itemize}
	
	\subsubsection{Audit-ability}
	\begin{itemize}
	\item Every action on the system must be recorded on a log that can later be viewed and queried.
	\item Actions that are to be recorded
	\begin{itemize}
	\item The current user performing the action.
	\item New mark capturing.
	\item Deletion of marks.
	\item Edition of marks.
	\item Reason for editing the marks.
	\item The date/time on which an action was performed.
	\end{itemize}
	\end{itemize}
	
\subsubsection{Scalabity}
	\begin{itemize}
	\item The system must be able to be accessed by up to the current number of students registered for the course at any point.
	\item Additions and updates must be available to all the TA's when a mark sheet is unlocked.
	\end{itemize}	
	
	\subsubsection{Authorization}
	\begin{itemize}
	\item All actions are granted to a user based on the privileges they have.
	\end{itemize}
	
	\subsubsection{Authentication}
	\begin{itemize}
	\item The system must have an access control mechanism.
	\item Access control must be handled at log in/out points.
	\end{itemize}
	
	
\subsection{Integration requirements}
\textbf{(NOT DONE - Who will do this?)}\\ \\



\subsection{Architectural constraints}
\textbf{(DONE by Thulasiswe - needs to be checked)}\\ \\
\subsubsection{Technologies to be used}
	\begin{itemize}
	\item For the application platform
	\begin{itemize}
	\item Android SDK
	\item Java
	\end{itemize}
	\item For the web interface
	\begin{itemize}
	\item JavaScript
	\item HTML 4.0/HTML 5
	\end{itemize}
	\item Must be secure therefore must run over HTTPS
	\item Server side
	\begin{itemize}
	\item Django framework
	\item Python
	\end{itemize}
	\item Web service
	\begin{itemize}
	\item SOAP
	\end{itemize}
	\item MySQL for database
	\item LDAP for distributed directory information service
	\end{itemize}
	
\section{Functional requirements}
\subsection{Introduction}
This section discusses the application functionality required by users (and other stakeholders).
\subsection{Scope and Limitations/Exclusions}
Use a high-level use case diagram with
\begin{itemize}
\item abstract use cases for services/responsibility domains of the system,
\item concrete use cases (the leaf use cases in the specialization hierarchy) as the required concrete use cases/user services,
\item optionally some include and extend relationships to show the core functional requirements, and
\item actors showing the external systems which are not part of the scope of the system, but which
the system integrates with.
\end{itemize}
List the exclusions/limitations, discussing any functionality which could erroneously be assumed within scope but which has been explicitly excluded from the scope of the system.
\subsection{Required functionality}
Use for each concrete use case a use case diagram with the required functionality in the form of includes and extends relationships to lower level use cases – this may be specified across levels of granularity.
\subsection{Use case prioritization}
Consider a simple three-level prioritization with
\\\\
\textbf{Critical: }A use case which is absolutely essential (ask whether the project should be cancelled if that functionality could not be provided).
\\\\
\textbf{Important: } The system would still be useful without some of the important use cases, but the client would get quantifiably less value from the system.
\\\\
\textbf{Nice-To-Have} Its a requirement but the value to the client/business is insignificant/not quantifiable.

\subsection{Use case/Services contracts}
For each use case/service specify
\\\\
\textbf{Pre-Conditions: }the conditions under which the service may be refused (usually there is an exception associated with each pre-condition).
\\\\
\textbf{Post-Conditions: }the conditions which must hold true after the service has been provided.
\\\\
\textbf{Request and Results Data Structures: }Use class diagrams to specify the data structure requirements for the request and result objects (i.e. the inputs and outputs).
\subsection{Process specifications}
For some of the use cases there may be requirements around the process which needs to be followed. If so, these requirements are typically specified via activity and/or sequence diagrams or
alternatively via state charts.
\subsection{Domain Objects}
Use UML class diagrams to specify the data structure requirements in a technology neutral way. These can ultimately be mapped onto different technologies like ERD diagrams/relational databases, XML schemas, Python/Java/C++/... objects, paper based or UI forms, ... But those are just different technology mappings and this would not be part of the requirements specification.
\section{Open Issues}
Discuss in this section
\begin{itemize}
\item any aspects of the requirements which still need to be specified,
\item around which clarification is still required, as well as
\item any discovered inconsistencies in the requirements.
\end{itemize}
\section{Glossary}
The requirements is to be read, understood and validated by a range of people from very different backgrounds (the client, domain experts/business analysts, the developers, software architects, users, ... ). Use a glossary to explain any terms which some parties may not be familiar with.
\end{document}