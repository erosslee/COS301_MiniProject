\documentclass[11pt,a4paper]{article}
\usepackage{hyperref}
\begin{document}
\begin{titlepage}
\title{Software Requirements Specification\\Computer Science Marks System\\ \small Version: 0.2\\ \url{https://github.com/erosslee/COS301_MiniProject}}
\author{Group 10: \\Estian Rosslee 12223426\\Moeletji Semenya 12349136\\Thulasizwe Mavuso 29236259\\Uteshlen Nadesan 28163304\\Zenadia Groenewald 12265676\\Z\"uhnja Riekert 12040593}
\maketitle
\end{titlepage}
\pagebreak
\tableofcontents
\pagebreak
\section{Introduction}
The requirements specification should ultimately contain sufficient information such that the system could be largely developed by a third party without further input. To this end the requirements must be precise and testable.
\\

The requirements need not be fully specified up-front. One might start with the vision, scope and architectural requirements, perform an upfront software architecture engineering phase and then iteratively elicit the detailed requirements for a use case, build, test and deploy the use case before adding the detailed requirements for the next use case. Such an approach follows solid engineering phase for the core software infrastructure/architecture with an agile software development approach within which the application functionality is developed iteratively.
\section{Vision}
A description of the vision of the project. This typically includes the main purpose of the project and what the client aims to achieve with the project.
\section{Background}
A general discussion of what lead to the project including potentially
\begin{itemize}
\item business/research opportunities,
\item opportunities to simplify/improve some aspect of life/work or community,
\item problems your client is currently facing,
\item ...
\end{itemize}
\section{Stakeholders}
\section{Architectural requirements}
\subsection{Access channel requirements}
Specify the different access channels through which the system's services are to be accessed by humans and by other systems (e.g. Mobile/Android application clients, Restful web services clients,
Browser clients, ... ).
\subsection{Quality requirements}
Specify and quantify each of the quality requirements which are relevant to the system. Examples of quality requirements include performance, reliability, scalability, security, flexibility, maintainability, auditability/monitorability, integrability, cost, usability. Each of these quality requirements need to be either quantified or at least be specified in a testable way.
\subsection{Integration requirements}
This section specifies any integration requirements for any external systems. This may include
\begin{itemize}
\item the integration channel to be used,
\item the protocols to be used,
\item API specifications in the form of UML interfaces and/or technology-specific API specifications
(e.g. WSDLs, CORBA IDLs, . . . ), and
\item any quality requirements for the integration itself (performance, scalability, reliability, security, auditability, ... ).
\end{itemize}
\subsection{Architectural constraints}
This specifies any constraints the client may specify on the system architecture include
\begin{itemize}
\item technologies which MUST be used,
\item architectural patterns/frameworks which must be used (e.g. layering, Services Oriented Architectures, ... )
\item ...
\end{itemize}
\section{Functional requirements}
\subsection{Introduction}
This section discusses the application functionality required by users (and other stakeholders).
\subsection{Scope and Limitations/Exclusions}
Use a high-level use case diagram with
\begin{itemize}
\item abstract use cases for services/responsibility domains of the system,
\item concrete use cases (the leaf use cases in the specialization hierarchy) as the required concrete use cases/user services,
\item optionally some include and extend relationships to show the core functional requirements, and
\item actors showing the external systems which are not part of the scope of the system, but which
the system integrates with.
\end{itemize}
List the exclusions/limitations, discussing any functionality which could erroneously be assumed within scope but which has been explicitly excluded from the scope of the system.
\subsection{Required functionality}
Use for each concrete use case a use case diagram with the required functionality in the form of includes and extends relationships to lower level use cases – this may be specified across levels of granularity.
\subsection{Use case prioritization}
Consider a simple three-level prioritization with
\\\\
\textbf{Critical: }A use case which is absolutely essential (ask whether the project should be cancelled if that functionality could not be provided).
\\\\
\textbf{Important: } The system would still be useful without some of the important use cases, but the client would get quantifiably less value from the system.
\\\\
\textbf{Nice-To-Have} Its a requirement but the value to the client/business is insignificant/not quantifiable.

\subsection{Use case/Services contracts}
For each use case/service specify
\\\\
\textbf{Pre-Conditions: }the conditions under which the service may be refused (usually there is an exception associated with each pre-condition).
\\\\
\textbf{Post-Conditions: }the conditions which must hold true after the service has been provided.
\\\\
\textbf{Request and Results Data Structures: }Use class diagrams to specify the data structure requirements for the request and result objects (i.e. the inputs and outputs).
\subsection{Process specifications}
For some of the use cases there may be requirements around the process which needs to be followed. If so, these requirements are typically specified via activity and/or sequence diagrams or
alternatively via state charts.
\subsection{Domain Objects}
Use UML class diagrams to specify the data structure requirements in a technology neutral way. These can ultimately be mapped onto different technologies like ERD diagrams/relational databases, XML schemas, Python/Java/C++/... objects, paper based or UI forms, ... But those are just different technology mappings and this would not be part of the requirements specification.
\section{Open Issues}
Discuss in this section
\begin{itemize}
\item any aspects of the requirements which still need to be specified,
\item around which clarification is still required, as well as
\item any discovered inconsistencies in the requirements.
\end{itemize}
\section{Glossary}
The requirements is to be read, understood and validated by a range of people from very different backgrounds (the client, domain experts/business analysts, the developers, software architects, users, ... ). Use a glossary to explain any terms which some parties may not be familiar with.
\end{document}