\section{Glossary}
\begin{itemize}

\item Android - Smart phone operating system to be programmed for
\item Authentication - to establish user as genuine
\item Authorization - permission or power granted by an authority
\item Client - a workstation on a network that gains access to central data files, programs, and peripheral devices through a server
\item COS - Computer Science (Courses)
\item Database - Web framework
\item Developers - The programmers of the application
\item End-users - Clients and/or the actual users of the interface
\item Evaluators - The markers in the practical session
\item Fitchfork - Auto marking system used by the Department of Computer Science
\item HTTPS - HyperText Transmission Protocol, Secure 
\item Interface - The part of the application that the end users will be using the application through
\item Java -  high-level, object-oriented computer programming language used especially to create interactive applications running over the Internet
\item JavaScript - an embedded language run in web browsers
\item LDAP - Lightweight Directory Access Protocol (Database of use for the Department of Computer Science)
\item MySQL - The most popular open source relational database management system
\item PDF - a file format that makes it possible to display text and graphics in the same fixed layout on any computer screen
\item Scalability - The ability of something, especially a computer system, to adapt to increased demands
\item SOAP - originally defined as Simple Object Access Protocol, is a protocol specification for exchanging structured information in the implementation of Web Services in computer networks


\end{itemize}
